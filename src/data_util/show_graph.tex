\documentclass[letter,10pt]{article}
\usepackage{amsmath}
\usepackage{amssymb}
\usepackage{amsthm}
\usepackage[retainorgcmds]{IEEEtrantools}
\usepackage{latexsym}
\usepackage{textcomp}
\usepackage{graphicx}
\usepackage{tikz}
\usetikzlibrary{shapes,graphs,graphdrawing,quotes,arrows}
\usegdlibrary{force, layered, trees}
\tikzset{>=latex}
\usepackage{enumerate}
\usepackage[top=1in, bottom=1in, left=1in, right=1in]{geometry}
\usepackage[pdftex]{hyperref}
\newtheorem{lemma}{Lemma}

%New commands
\newcommand{\ud}{\,\mathrm{d}}
\newcommand{\la}{\Rightarrow}
\newcommand{\lr}{\Longleftrightarrow}
\newcommand{\rfe}[1]{(\ref{eqn:#1})}

\begin{document}
\title{\textbf{Show Me the Graph! --- Automatic Graph Generator}}
\author{Larry Yihe Tang}
\date{\today}
\maketitle


% Use tree layout for trees.
\[ \begin{tikzpicture}
\graph[nodes={draw,circle}, layered layout, node distance=2.5cm]  {

% Paste generated code below this line.
"/\textbackslash:c0" -> {"=:c1","=:c8"},
"=:c1" -> {"mapAt:c2","xs7"[rectangle]},
"mapAt:c2" -> {"NIL:c3","f5"[rectangle],"xs7"[rectangle]},
"NIL:c3" -> {},
"!!4"[diamond] -> {"/\textbackslash:c0","f20\#F"[rectangle],"f5"[rectangle]},
"f5"[rectangle] -> {},
"!!6"[diamond] -> {"!!4"[diamond],"xs7"[rectangle]},
"xs7"[rectangle] -> {},
"=:c8" -> {"mapAt:c9","COND:c15"},
"mapAt:c9" -> {"CONS:c10","f5"[rectangle],"xs7"[rectangle]},
"CONS:c10" -> {"n12"[rectangle],"ns14"[rectangle]},
"!!11"[diamond] -> {"!!6"[diamond],"n12"[rectangle]},
"n12"[rectangle] -> {},
"!!13"[diamond] -> {"!!11"[diamond],"ns14"[rectangle]},
"ns14"[rectangle] -> {},
"COND:c15" -> {"$<$:c16","mapAt:c18","mapAt:c22"},
"$<$:c16" -> {"n12"[rectangle],"LENGTH:c17"},
"LENGTH:c17" -> {"xs7"[rectangle]},
"mapAt:c18" -> {"ns14"[rectangle],"f5"[rectangle],"list\_update:c19"},
"list\_update:c19" -> {"xs7"[rectangle],"n12"[rectangle],"f20\#F"[rectangle]},
"f20\#F"[rectangle] -> {"\#!\#:c21"},
"\#!\#:c21" -> {"xs7"[rectangle],"n12"[rectangle]},
"mapAt:c22" -> {"ns14"[rectangle],"f5"[rectangle],"xs7"[rectangle]},
"$\mid$-:c23" -> {"!!13"[diamond]},
% Paste generated code above this line.

};
\end{tikzpicture} \]


\end{document}
